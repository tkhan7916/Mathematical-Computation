% Options for packages loaded elsewhere
\PassOptionsToPackage{unicode}{hyperref}
\PassOptionsToPackage{hyphens}{url}
%
\documentclass[
]{article}
\usepackage{lmodern}
\usepackage{amssymb,amsmath}
\usepackage{ifxetex,ifluatex}
\ifnum 0\ifxetex 1\fi\ifluatex 1\fi=0 % if pdftex
  \usepackage[T1]{fontenc}
  \usepackage[utf8]{inputenc}
  \usepackage{textcomp} % provide euro and other symbols
\else % if luatex or xetex
  \usepackage{unicode-math}
  \defaultfontfeatures{Scale=MatchLowercase}
  \defaultfontfeatures[\rmfamily]{Ligatures=TeX,Scale=1}
\fi
% Use upquote if available, for straight quotes in verbatim environments
\IfFileExists{upquote.sty}{\usepackage{upquote}}{}
\IfFileExists{microtype.sty}{% use microtype if available
  \usepackage[]{microtype}
  \UseMicrotypeSet[protrusion]{basicmath} % disable protrusion for tt fonts
}{}
\makeatletter
\@ifundefined{KOMAClassName}{% if non-KOMA class
  \IfFileExists{parskip.sty}{%
    \usepackage{parskip}
  }{% else
    \setlength{\parindent}{0pt}
    \setlength{\parskip}{6pt plus 2pt minus 1pt}}
}{% if KOMA class
  \KOMAoptions{parskip=half}}
\makeatother
\usepackage{xcolor}
\IfFileExists{xurl.sty}{\usepackage{xurl}}{} % add URL line breaks if available
\IfFileExists{bookmark.sty}{\usepackage{bookmark}}{\usepackage{hyperref}}
\hypersetup{
  pdftitle={Matrix Norms},
  hidelinks,
  pdfcreator={LaTeX via pandoc}}
\urlstyle{same} % disable monospaced font for URLs
\usepackage[margin=1in]{geometry}
\usepackage{color}
\usepackage{fancyvrb}
\newcommand{\VerbBar}{|}
\newcommand{\VERB}{\Verb[commandchars=\\\{\}]}
\DefineVerbatimEnvironment{Highlighting}{Verbatim}{commandchars=\\\{\}}
% Add ',fontsize=\small' for more characters per line
\usepackage{framed}
\definecolor{shadecolor}{RGB}{248,248,248}
\newenvironment{Shaded}{\begin{snugshade}}{\end{snugshade}}
\newcommand{\AlertTok}[1]{\textcolor[rgb]{0.94,0.16,0.16}{#1}}
\newcommand{\AnnotationTok}[1]{\textcolor[rgb]{0.56,0.35,0.01}{\textbf{\textit{#1}}}}
\newcommand{\AttributeTok}[1]{\textcolor[rgb]{0.77,0.63,0.00}{#1}}
\newcommand{\BaseNTok}[1]{\textcolor[rgb]{0.00,0.00,0.81}{#1}}
\newcommand{\BuiltInTok}[1]{#1}
\newcommand{\CharTok}[1]{\textcolor[rgb]{0.31,0.60,0.02}{#1}}
\newcommand{\CommentTok}[1]{\textcolor[rgb]{0.56,0.35,0.01}{\textit{#1}}}
\newcommand{\CommentVarTok}[1]{\textcolor[rgb]{0.56,0.35,0.01}{\textbf{\textit{#1}}}}
\newcommand{\ConstantTok}[1]{\textcolor[rgb]{0.00,0.00,0.00}{#1}}
\newcommand{\ControlFlowTok}[1]{\textcolor[rgb]{0.13,0.29,0.53}{\textbf{#1}}}
\newcommand{\DataTypeTok}[1]{\textcolor[rgb]{0.13,0.29,0.53}{#1}}
\newcommand{\DecValTok}[1]{\textcolor[rgb]{0.00,0.00,0.81}{#1}}
\newcommand{\DocumentationTok}[1]{\textcolor[rgb]{0.56,0.35,0.01}{\textbf{\textit{#1}}}}
\newcommand{\ErrorTok}[1]{\textcolor[rgb]{0.64,0.00,0.00}{\textbf{#1}}}
\newcommand{\ExtensionTok}[1]{#1}
\newcommand{\FloatTok}[1]{\textcolor[rgb]{0.00,0.00,0.81}{#1}}
\newcommand{\FunctionTok}[1]{\textcolor[rgb]{0.00,0.00,0.00}{#1}}
\newcommand{\ImportTok}[1]{#1}
\newcommand{\InformationTok}[1]{\textcolor[rgb]{0.56,0.35,0.01}{\textbf{\textit{#1}}}}
\newcommand{\KeywordTok}[1]{\textcolor[rgb]{0.13,0.29,0.53}{\textbf{#1}}}
\newcommand{\NormalTok}[1]{#1}
\newcommand{\OperatorTok}[1]{\textcolor[rgb]{0.81,0.36,0.00}{\textbf{#1}}}
\newcommand{\OtherTok}[1]{\textcolor[rgb]{0.56,0.35,0.01}{#1}}
\newcommand{\PreprocessorTok}[1]{\textcolor[rgb]{0.56,0.35,0.01}{\textit{#1}}}
\newcommand{\RegionMarkerTok}[1]{#1}
\newcommand{\SpecialCharTok}[1]{\textcolor[rgb]{0.00,0.00,0.00}{#1}}
\newcommand{\SpecialStringTok}[1]{\textcolor[rgb]{0.31,0.60,0.02}{#1}}
\newcommand{\StringTok}[1]{\textcolor[rgb]{0.31,0.60,0.02}{#1}}
\newcommand{\VariableTok}[1]{\textcolor[rgb]{0.00,0.00,0.00}{#1}}
\newcommand{\VerbatimStringTok}[1]{\textcolor[rgb]{0.31,0.60,0.02}{#1}}
\newcommand{\WarningTok}[1]{\textcolor[rgb]{0.56,0.35,0.01}{\textbf{\textit{#1}}}}
\usepackage{graphicx}
\makeatletter
\def\maxwidth{\ifdim\Gin@nat@width>\linewidth\linewidth\else\Gin@nat@width\fi}
\def\maxheight{\ifdim\Gin@nat@height>\textheight\textheight\else\Gin@nat@height\fi}
\makeatother
% Scale images if necessary, so that they will not overflow the page
% margins by default, and it is still possible to overwrite the defaults
% using explicit options in \includegraphics[width, height, ...]{}
\setkeys{Gin}{width=\maxwidth,height=\maxheight,keepaspectratio}
% Set default figure placement to htbp
\makeatletter
\def\fps@figure{htbp}
\makeatother
\setlength{\emergencystretch}{3em} % prevent overfull lines
\providecommand{\tightlist}{%
  \setlength{\itemsep}{0pt}\setlength{\parskip}{0pt}}
\setcounter{secnumdepth}{-\maxdimen} % remove section numbering

\title{Matrix Norms}
\author{}
\date{\vspace{-2.5em}}

\begin{document}
\maketitle

A ``matrix norm'' is a way of assigning a numerical measurement to a
matrix. There are different types of matrix norms, each useful in their
own context.

\begin{center}\rule{0.5\linewidth}{0.5pt}\end{center}

The \emph{Frobenius norm} is defined as follows:

\hypertarget{a_f-sqrtdisplaystyle-sum_i-1nsum_j1ma_ij2}{%
\paragraph{\texorpdfstring{\[||A||_{F} = \sqrt{\displaystyle \sum_{i = 1}^{n}\sum_{j=1}^{m}(A_{i,j})^2}\]}{\textbar\textbar A\textbar\textbar\_\{F\} = \textbackslash sqrt\{\textbackslash displaystyle \textbackslash sum\_\{i = 1\}\^{}\{n\}\textbackslash sum\_\{j=1\}\^{}\{m\}(A\_\{i,j\})\^{}2\}}}\label{a_f-sqrtdisplaystyle-sum_i-1nsum_j1ma_ij2}}

The function for the \emph{Frobenius norm} is written in R as follows:

\begin{Shaded}
\begin{Highlighting}[]
\NormalTok{F.norm \textless{}{-}}\StringTok{ }\ControlFlowTok{function}\NormalTok{(A) \{}
\NormalTok{  sum\_squares =}\StringTok{ }\DecValTok{0}
  \ControlFlowTok{for}\NormalTok{ (i }\ControlFlowTok{in} \DecValTok{1}\OperatorTok{:}\KeywordTok{nrow}\NormalTok{(A)) \{}
    \ControlFlowTok{for}\NormalTok{ (j }\ControlFlowTok{in} \DecValTok{1}\OperatorTok{:}\KeywordTok{ncol}\NormalTok{(A)) \{}
\NormalTok{      sum\_squares =}\StringTok{ }\NormalTok{sum\_squares }\OperatorTok{+}\StringTok{ }\NormalTok{A[i,j]}\OperatorTok{\^{}}\DecValTok{2}
\NormalTok{    \}}
\NormalTok{  \}}
  \KeywordTok{return}\NormalTok{(}\KeywordTok{sqrt}\NormalTok{(sum\_squares))}
\NormalTok{\}}
\end{Highlighting}
\end{Shaded}

The \emph{1-norm} is defined as follows:

\hypertarget{a_1-displaystyle-max_1-le-j-le-msum_i1na_ij}{%
\paragraph{\texorpdfstring{\[||A||_{1} = \displaystyle \max_{1 \le j \le m}(\sum_{i=1}^{n}|A_{i,j}|)\]}{\textbar\textbar A\textbar\textbar\_\{1\} = \textbackslash displaystyle \textbackslash max\_\{1 \textbackslash le j \textbackslash le m\}(\textbackslash sum\_\{i=1\}\^{}\{n\}\textbar A\_\{i,j\}\textbar)}}\label{a_1-displaystyle-max_1-le-j-le-msum_i1na_ij}}

The function for the \emph{1-norm} is written in R as follows:

\begin{Shaded}
\begin{Highlighting}[]
\NormalTok{one.norm \textless{}{-}}\StringTok{ }\ControlFlowTok{function}\NormalTok{(A) \{}
\NormalTok{  n =}\StringTok{ }\KeywordTok{nrow}\NormalTok{(A)}
\NormalTok{  m =}\StringTok{ }\KeywordTok{ncol}\NormalTok{(A)}
\NormalTok{  col\_sums =}\StringTok{ }\KeywordTok{vector}\NormalTok{(}\DataTypeTok{length=}\NormalTok{m)}
  \ControlFlowTok{for}\NormalTok{ (j }\ControlFlowTok{in} \DecValTok{1}\OperatorTok{:}\NormalTok{m) \{}
\NormalTok{    sum =}\StringTok{ }\DecValTok{0}
    \ControlFlowTok{for}\NormalTok{ (i }\ControlFlowTok{in} \DecValTok{1}\OperatorTok{:}\NormalTok{n) \{}
\NormalTok{      sum =}\StringTok{ }\NormalTok{sum }\OperatorTok{+}\StringTok{ }\KeywordTok{abs}\NormalTok{(A[i,j])}
\NormalTok{    \}}
\NormalTok{    col\_sums[j] =}\StringTok{ }\NormalTok{sum}
\NormalTok{  \}}
  \KeywordTok{return}\NormalTok{(}\KeywordTok{max}\NormalTok{(col\_sums))}
\NormalTok{\}}
\end{Highlighting}
\end{Shaded}

The \emph{\(\infty\)-norm} is defined as follows:

\hypertarget{a_infty-displaystyle-max_1-le-i-le-nsum_j1na_ij}{%
\paragraph{\texorpdfstring{\[||A||_{\infty} = \displaystyle \max_{1 \le i \le  n}(\sum_{j=1}^{n}|A_{i,j}|)\]}{\textbar\textbar A\textbar\textbar\_\{\textbackslash infty\} = \textbackslash displaystyle \textbackslash max\_\{1 \textbackslash le i \textbackslash le  n\}(\textbackslash sum\_\{j=1\}\^{}\{n\}\textbar A\_\{i,j\}\textbar)}}\label{a_infty-displaystyle-max_1-le-i-le-nsum_j1na_ij}}

The function for the \emph{\(\infty\)-norm} is written in R as follows:

\begin{Shaded}
\begin{Highlighting}[]
\NormalTok{inf.norm \textless{}{-}}\StringTok{ }\ControlFlowTok{function}\NormalTok{(A) \{}
\NormalTok{  n =}\StringTok{ }\KeywordTok{nrow}\NormalTok{(A)}
\NormalTok{  m =}\StringTok{ }\KeywordTok{ncol}\NormalTok{(A)}
\NormalTok{  row\_sums =}\StringTok{ }\KeywordTok{vector}\NormalTok{(}\DataTypeTok{length=}\NormalTok{n)}
  \ControlFlowTok{for}\NormalTok{ (i }\ControlFlowTok{in} \DecValTok{1}\OperatorTok{:}\NormalTok{n) \{}
\NormalTok{    sum =}\StringTok{ }\DecValTok{0}
    \ControlFlowTok{for}\NormalTok{ (j }\ControlFlowTok{in} \DecValTok{1}\OperatorTok{:}\NormalTok{m) \{}
\NormalTok{      sum =}\StringTok{ }\NormalTok{sum }\OperatorTok{+}\StringTok{ }\KeywordTok{abs}\NormalTok{(A[i,j])}
\NormalTok{    \}}
\NormalTok{    row\_sums[i] =}\StringTok{ }\NormalTok{sum}
\NormalTok{  \}}
  \KeywordTok{return}\NormalTok{(}\KeywordTok{max}\NormalTok{(row\_sums))}
\NormalTok{\}}
\end{Highlighting}
\end{Shaded}

\begin{center}\rule{0.5\linewidth}{0.5pt}\end{center}

The main Matrix Norm function is written in R as follows:

\begin{Shaded}
\begin{Highlighting}[]
\NormalTok{mat.norm \textless{}{-}}\StringTok{ }\ControlFlowTok{function}\NormalTok{(A, }\DataTypeTok{type=}\KeywordTok{c}\NormalTok{(}\StringTok{"one"}\NormalTok{, }\StringTok{"inf"}\NormalTok{, }\StringTok{"F"}\NormalTok{)) \{}
  \ControlFlowTok{switch}\NormalTok{ (type,}
    \StringTok{"one"}\NormalTok{ =}\StringTok{ }\KeywordTok{one.norm}\NormalTok{(A),}
    \StringTok{"inf"}\NormalTok{ =}\StringTok{ }\KeywordTok{inf.norm}\NormalTok{(A),}
    \StringTok{"F"}\NormalTok{ =}\StringTok{ }\KeywordTok{F.norm}\NormalTok{(A)}
\NormalTok{  )}
\NormalTok{\}}
\end{Highlighting}
\end{Shaded}

\begin{center}\rule{0.5\linewidth}{0.5pt}\end{center}

Homework 5 for MATH 366: \emph{Applied Mathematical Computation}

\end{document}
